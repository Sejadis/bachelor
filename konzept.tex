\chapter{Konzept}
\section{Struktur Nodes}
Um eine einfache sowie übersichtliche Steuerung des Greifarms zu ermöglichen wird die Funktionalität in mehrere ROS2 Nodes aufgeteilt. Generell werden folgende Funktionalitäten benötigt: der Planer, die Speicherung des aktuellen Zustands der Welt, die Ausführung des Plans sowie die Möglichkeit Eingaben zu verarbeiten und an die entsprechende Stelle weiterzuleiten.\newline
Diese Aufteilung entspricht auch einer guten Aufteilung und Trennung der Verantwortungen um daraus ROS2 Nodes zu machen. Hier haben die Nodes folgende Verantwortlichkeiten:\newline
Die Planungs-Node muss mit einer gegebenen Domäne und einem Problem eine einen Plan bestehend aus einer Reihe von Aktionen zurückgeben.\newline
Die Welt-Node hält den aktuellen Zustand der Welt bzw. des Problems und muss diesen konsistent halten.\newline
Die Ausführungs-Node muss entsprechend eines gegebenen Plans die beinhalteten Aktionen ausführen.\newline
Die Eingabe-Node ermöglicht die Erstellung eines Problem mit einem Welt-Zustand und einem Ziel.