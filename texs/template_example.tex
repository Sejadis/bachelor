
\section{Abschnitt} 
Hier soll eine kurze Einführung erfolgen, die den Zusammenhang des Kapitels zur Arbeit herstellt.

Generelle Hinweise:

\begin{itemize}
\item eindeutige Begrifflichkeit verwenden
\item auf logische Herleitung der Argumentation achten
\end{itemize}

Ein Hauptabschnitt - idealerweise sollten keine Abschnitte leer sein.

\subsection{Unterabschnitt}
Ein Unterabschnitt - idealerweise sollten keine Abschnitte leer sein.

\subsubsection{Unterunterabschnitt}
Ein Unterunterabschnitt - idealerweise sollten keine Abschnitte leer sein.

\section{Einfache Formatvorlagen}

\textbf{Das ist fett gedruckter Text}.

\textit{Das ist kursiver Text}.


Auflistungen sind oft hilfreich für die Strukturierung:
\begin{itemize}
    \item Erster Eintrag
    \item Zweiter Eintrag
\end{itemize}

Nummerierte Aufzählungen sind oft hilfreich für Reihenfolgen:
\begin{enumerate}
    \item Erster Eintrag
    \item Zweiter Eintrag
\end{enumerate}


\section{Zitieren und Referenzieren}

Beiträge in Fachzeitschriften wie \citet[12]{clemen1989combining} oder Konferenzartikel wie \citet[6]{he2017mask} werden auf diese Weise im Text zitiert. In anderen Fällen möchte man aber in Klammern zitieren \citep[10]{clemen1989combining}, auch mit mehreren Autoren (\citealp[3]{baumol1958warehouse}; \citealp[15]{clemen1989combining}; \citealp[12]{he2017mask}).

Die Seitenzahlen müssen angegeben werden \citep[28]{chollet2018deep}. Bezieht sich das Zitat auf eine Textstelle, die sich über mehrere Seiten streckt, so sind diese entsprechend anzugeben: \citet[28-29]{chollet2018deep} bzw. \citet[28-35]{chollet2018deep}

So wird eine Webquelle zitiert: \citet{shiny1}. Es kann bei kurzen Informationen im Internet aber auch reichen die Adresse\footnote{\url{https://shiny.rstudio.com/tutorial/written-tutorial/lesson1/}} als Fußnote einzubetten.

Bei einem direkten Zitat muss der zitierte Text originalgetreu wiedergegeben werden. Rechtschreibfehler oder eine veraltete Orthographie werden unverändert wiedergegeben. „Der zitierte Text steht immer in Anführungszeichen"\ \citep[28]{chollet2018deep}. 

So werden andere Teile der Arbeit referenziert: Kapitel \ref{einleitung}, Gleichung \ref{eq:1} zeigt...

So verweisen wir auf eine Fußnote \footnote{dies ist eine Fußnote}.

\section{Abbildungen}

Abbildungen erfordern das package \textit{graphicx}. 
Idealerweise verwendet man Vektorgrafiken oder hochaufgelöste Bitmaps. 
Eine gute Variante ist das Verwenden von PDFs.

\begin{figure}[h]
    \centering
    \includegraphics[width=0.3\textwidth]{figs/2015_10_05_THB_FB-IM_Logo_RGB.pdf}
    \caption{Siegel der Universität}
    \label{fig:my_label}
\end{figure}


\section{Tabellen}

Die Tabular-Umgebung gibt die Anzahl Spalten an, deren Orientierung, Breite und evtl. Zwischenlinien. 


\begin{table}[ht]
    \centering
    \caption{Meine Tabelle}
        \begin{tabular}{ cccc } 
        \toprule
        col1 & col2 & col3 \\
        \midrule
        \multirow{3}{4em}{Multiple row} & cell2 & cell3 \\ 
        & cell5 & cell6 \\ 
        & cell8 & cell9 \\ 
        \bottomrule
    \end{tabular}
    \label{tab:countries}
\end{table}

\section{Formeln}

\begin{equation}
    \sum_{i=1}^N x_i
    \label{eq:1}
\end{equation}

\section{Abkürzungen}
Bei der ersten Verwendung wird die Abkürzung eines Fachbegriffs wie zum Beispiel \ac{ML} eingeführt und daher ausgeschrieben. Bei der zweiten Verwendung der Abkürzung \ac{ML} ist dies nicht mehr nötig. Die Abkürzungen sind in dem Abschnitt \textit{Definition der Abkürzungen} einzupflegen. Das Abkürzungsverzeichnis ist alphabetisch anzuordnen. Bleibt das Abkürzungsverzeichnis leer, so kann dieser Abschnitt (Zeilen 164-173 im main.tex) gelöscht werden.


\section{Fazit}
In der Schlussfolgerung sollen

\begin{itemize}
\item die Themenstellung
\item der gewählte Ansatz
\item die Ergebnisse der Arbeit
\item eine kritische Stellungnahme/Einschätzung
\item nächste Schritte
\end{itemize}
deutlich werden.

Hinweis:
Die Schlussfolgerung sollte mit der Zusammenfassung bzw. dem Abstract und der Einleitung abgeglichen werden. Es sollte immer eine Zusammenfassung der wesentlichen Er-kenntnisse der eigenen Arbeit sein, die den Forschungsbeitrag darstellt. Der Umfang der Schlussfolgerung sollte ähnlich wie die Einleitung ca. 5 \% der gesamten Arbeit betragen.

