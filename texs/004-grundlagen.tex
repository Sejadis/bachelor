\section {Grundlagen}
\subsection{ROS2}
\ac{ROS2} ist eine Sammlung von Bibliotheken und Werkzeugen für Robotik-Applikationen welche alle OpenSource sind.
Die erste \ac{ROS2} Release-Version erschien im Dezember 2017 unter dem Namen Ardent Apalon \citet{ros2docs}.\\
Es werden  mehrere \acp{RCL}  zur Verfügung gestellt, welche den Zugriff auf die \ac{ROS2}-API ermöglichen.
Die \acp{RCL} für die Sprachen C++ und Python (rclcpp und rclpy) werden dabei direkt vom \ac{ROS2}-Team verwaltet.
Von der Community wurden weitere \acp{RCL}, unter anderem für die Sprachen C\#, Swift und Rust, entwickelt.\\
Um die Entwicklung der \acp{RCL} zu vereinfachen und die Logik sprachen unabhängig zu machen werden Funktionalitäten als C Interfaces zugänglich gemacht, für welche in den \acp{RCL} Wrapper geschrieben werden.
\subsubsection{Node}
\subsection{ROS2 Interfaces}
In diesem Abschnitt werden die verschiedenen Arten von Interfaces beschrieben, welche für die im folgenden Abschnitt~\ref{rosnodecomm} erklärten Methoden zur Kommunikation zwischen mehreren Nodes genutzt werden.\\
\ac{ROS2} Interfaces sind Definitionen, welche Daten mit welchen Typen gesendet werden.
Sie werden in der \ac{IDL} geschrieben, welche es ermöglicht automatisch Source Code in verschiedenen Sprachen für diese zu generieren.\\
- Erklären was für den build nötig ist?
\subsubsection{Message}
Eine Message ist der einfachste Typ der \ac{ROS2} Interfaces, welcher gleichzeitig auch ein Baustein für die folgenden Interfaces sein wird.\\
Message Definitionen haben die Dateiendung \verb|.msg| und werden per Konvention in einem Ordner \verb|msg/| gespeichert.
Jede \verb|.msg| Datei besteht aus den folgenden Teilen:
\begin{itemize}
\item Felder
\item Konstanten
\end{itemize}
Jedes Feld besteht aus einem Typ und einem durch ein Leerzeichen getrennten Namen: \verb|typ name|, z.B. \verb|bool isDone|.
Als Typ können dabei die integrierten Typen (s. Tabelle~\ref{tab:builtintypes}) oder der Name anderer Messages genutzt werden.
Zusätzlich kann jeder integrierte Typ als Array definiert werden (s. Tabelle~\ref{tab:arraytypes}).\\
Feldnamen müssen kleingeschrieben sein.
Es können alphanumerische Zeichen sowie der Unterstrich zur Trennung von Wörtern genutzt werden.
Das erste Zeichen muss ein Buchstabe sein und der Name darf nicht mit einem Unterstrich enden.
Zusätzlich darf es keine aufeinander folgenden Unterstriche geben.\\
Konstanten behalten das Format der Felder bei.
Der Name der Konstante wird komplett in Großbuchstaben geschrieben.
Zusätzlich bekommen Konstanten einen Wert zugewiesen, welcher nicht innerhalb des Programms geändert werden kann.
Die Zuweisung des Wertes erfolgt mit dem \verb|=| Zeichen: \verb|string EXAMPLE="test"|.
\begin{table}[ht]
    \centering
    \caption{Integrierte Datentypen für Interfaces und deren C++ Equivalent}
\begin{tabular}{|l|l|}
\hline
\textbf{Typ} & \textbf{C++}   \\ \hline
bool         & bool           \\ \hline
byte         & uint8\_t       \\ \hline
char         & char           \\ \hline
float32      & float          \\ \hline
float64      & double         \\ \hline
int8         & int8\_t        \\ \hline
uint8        & uint8\_t       \\ \hline
int16        & int16\_t       \\ \hline
uint16       & uint16\_t      \\ \hline
int32        & int32\_t       \\ \hline
uint32       & uint32\_t      \\ \hline
int64        & int64\_t       \\ \hline
uint64       & uint64\_t      \\ \hline
string       & std::string    \\ \hline
wstring      & std::u16string \\ \hline
\end{tabular}
    \label{tab:builtintypes}
\end{table}
\begin{table}[ht]
    \centering
    \caption{Array Typen und deren C++ Equivalent}
\begin{tabular}{|l|l|}
\hline
\textbf{Typ} & \textbf{C++}   \\ \hline
static array               & std::array<T, N>   \\ \hline
unbounded dynamic array    & std::vector        \\ \hline
bounded dynamic array      & custom\_class<T, N> \\ \hline
bounded string             & std::string        \\ \hline
\end{tabular}
    \label{tab:arraytypes}
\end{table}
\subsubsection{Service}
Service Definitionen beschreiben eine Anfrage und eine Antwort.
Die Definition hat die Dateiendung \verb|.srv| und wird im Ordner \verb|srv/| gespeichert.
Anfrage und Antwort werden innerhalb der Datei durch \verb|-| getrennt.
Für beide Teile gilt, dass sie gültig sind, wenn sie einer gültigen Message Definition entsprechen.
Ein Beispiel einer einfachen Servicedefinition ist in Listing~\ref{lst:serviceexample} zu sehen.\\
\begin{minipage}{\linewidth}%minipage to prevent page break
\begin{lstlisting}[caption={Beispiel einer Service Definition}, label={lst:serviceexample}]
int32 request_int
string request_string
---
float32 response_float
\end{lstlisting}
\end{minipage}
\subsubsection{Action}
Action Definitionen bestehen aus den 3 Teilen Anfrage, Ergebnis und Feedback, in dieser Reihenfolge.
Wie auch für Service Definitionen gilt, das die Teile durch \verb|-| getrennt sind und jeder einzelne Teil gültig ist, wenn er einer gültigen Message Definition entspricht.
\subsection{ROS2 Node Kommunikation}{\label{rosnodecomm}}
Damit verschiedene Nodes untereinander kommunizieren können, gibt es verschieden Mechanismen, welche sich primär darin unterscheiden, in welche Richtungen Nachrichten gesendet werden und ob es eine direkte Reaktion gibt.
Für alle Mechanismen gilt, das sie von einer Node unter einem bestimmten Namen sowie einem Typ zur Verfügung gestellt werden und von anderen Nodes durch eben diese genutzt werden können.
\subsubsection{Topic}
Topics entsprechen in etwa einem Newsletter System: eine Node veröffentlicht Daten, welche von allen anderen Nodes empfangen wird, die sich für diese registriert haben.
Das Format der Daten entspricht einer gewählten Message Definition.
\subsubsection{Service und Client}
Services werden für Prozesse genutzt, in denen eine Anfrage gesendet und eine Antwort erwartet wird.
Als Typ wird eine Servicedefinition genutzt.
\subsubsection{Action Server und Client}
Actions sind für länger andauernde Prozesse gedacht.
Eine Node erstellt einen Action Server über welchen eine Action zur Verfügung gestellt wird.
Andere Nodes können über einen Action Client eine Anfrage senden.
Der Server bearbeitet die Anfrage und sendet bei Beendigung eine Antwort mit einem Ergebnis an den Client.
Während der Dauer der Action kann der Server den Client optional mit Feedback Nachrichten über den aktuellen Status informieren.\\
Als Typ wird eine Action Definition genutzt. 
\subsection{OpenMANIPULATOR-X}
Der OpenMANIPULATOR-X ist ein von der Firma ROBOTIS{\footnote{http://en.robotis.com}} nach den Prinzipien ``OpenSoftware`` und ``OpenHardware`` hergestellter Greifarm.
OpenSoftware steht hierbei dafür, dass es ein OpenSource Projekt ist und auf dem OpenSource Projekt \ac{ROS2} basiert.
OpenHardware steht dafür, dass die meisten Komponenten als STL-Dateien zur Verfügung stehen und als Ersatzteile oder zum Anpassen des Greifarms mittels eines 3D-Druckers selbst hergestellt werden können.\\
Der OMX(Greifarm?,Abk?) ist eine 5DOF (5 Degrees of Freedom) Plattform, welche mittels 5 Servomotoren{\footnote{DYNAMIXEL XM430-W350-T}} gesteuert wird.
Dies ist aufgeteilt in 4DOF für den Arm sowie 1DOF für den Greifer.
Es kann eine Last bis 500g getragen werden
\subsection {Planung}
\subsubsection{STRIPS}
\subsubsection{Planning Domain Definition Language}
Die \ac{PDDL} ist eine Sprache zur Modellierung von Planungs...
\subsubsection{PDDL-Plugin für VS Code}
- nach Implementierung verschieben?\\
- Installation\\
- Nutzung / Ausführung\\
- Pfad zur popf executable im foxy ordner\\
- Screenshot mit Bsp. für Planausgabe und Visualisierung\\
\subsubsection{Partial Order Planning}
\subsubsection{Forward Chaining Partial Order Planning}
\subsubsection{Partial Order Planning Forward}
\citep{popf}
\subsubsection{Behavior Tree}