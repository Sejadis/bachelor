\section {Grundlagen}
\subsection{ROS2}
\ac{ROS2} ist eine Sammlung von Bibliotheken und Werkzeugen für Robotik-Applikationen welche alle OpenSource sind. Die erste \ac{ROS2} Release-Version erschien im Dezember 2017 unter dem Namen Ardent Apalon \citet{ros2docs}.\\
Es werden  mehrere \acp{RCL}  zur Verfügung gestellt, welche den Zugriff auf die \ac{ROS2}-API ermöglichen. Die \acp{RCL} für die Sprachen C++ und Python (rclcpp und rclpy) werden dabei direkt vom \ac{ROS2}-Team verwaltet. Von der Community wurden weitere \acp{RCL}, unter anderem für die Sprachen C\#, Swift und Rust, entwickelt.\\
Um die Entwicklung der \acp{RCL} zu vereinfachen und die Logik sprachenunabhängig zu machen werden Funktionalitäten als C Interfaces zugängllich gemacht, für welche in den \acp{RCL} Wrapper geschrieben werden.
\subsubsection{Node}
\subsubsection{Message}
\subsubsection{Service}
\subsubsection{Action}
\subsubsection{Topic}
\subsection{OpenMANIPULATOR-X}
Der OpenMANIPULATOR-X ist ein von der Firma ROBOTIS{\footnote{http://en.robotis.com}} nach den Prinzipien ``OpenSoftware`` und ``OpenHardware`` hergestellter Greifarm. OpenSoftware steht hierbei dafür, dass es ein OpenSource Projekt ist und auf dem OpenSource Projekt \ac{ROS2} basiert. OpenHardware steht dafür, dass die meisten Komponenten als STL-Dateien zur Verfügung stehen und als Ersatzteile oder zum Anpassen des Greifarms mittels eines 3D-Druckers selbst hergestellt werden können. \newline
Der OMX(Greifarm?,Abk?) ist eine 5DOF (5 Degrees of Freedom) Plattform, welche mittels 5 Servomotoren{\footnote{DYNAMIXEL XM430-W350-T}} gesteuert wird. Dies ist aufgeteilt in 4DOF für den Arm sowie 1DOF für den Greifer.
Es kann eine Last bis 500g getragen werden
\subsection {Planung}
\subsubsection{STRIPS}
\subsubsection{PDDL}
\subsubsection{PDDL-Plugin für VS Code}
\subsubsection{Partial Order Planning}
\subsubsection{Partial Order Planning Forward}
\citep{popf}
\subsubsection{Behavior Tree}

\newpage