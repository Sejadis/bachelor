\section{Ergebnis}
\subsection{Zusammenfassung}
- off. Unterstützung von Foxy des OMX erst während der Bearbeitung
Der Zusammenbau erfolgte ohne größere Probleme.
Die beschriebenen Details, die leicht zu übersehen sind, lassen sich auch im Nachhinein ohne großen Aufwand beheben oder korrigieren.
Durch die verfügbaren Installations-Skripte gab es auch Softwareseitig keine Hindernisse für einen schnellen Einstieg.
Auch wenn zu Beginn dieser Arbeit noch keine offizielle Unterstützung von \ac{ROS2} Foxy vorhanden war, konnte anhand der Anleitung für ältere \ac{ROS2} Distributionen alles zum laufen gebracht werden.
Durch die Kombination der \ac{ROS2} Foxy Tutorials sowie der OMX API Dokumentation ließen sich auch ohne Vorkenntnisse gute Fortschritte machen.
\subsection{Ausblick}
\subsubsection{MoveIt}
Um den in der dieser Arbeit verwendeten Ansatz der Bewegungsplanung robuster und dynamischer zu machen empfiehlt sich die Implementierung der Bibliothek MoveIt \footnote{MOVEIT URL}.
Zum Zeitpunkt der Bearbeitung wird diese vom OMX beim Betrieb mit \ac{ROS2} Foxy allerdings noch nicht offiziell unterstützt.\\
MoveIt bietet mittels virtueller Szenen die Möglichkeit effiziente und kollisionsfreie Wege für die Bewegung des OMX zu planen und auszuführen.
In der virtuellen Szene befindet sich sowohl ein Modell des Roboters als auch aller relevanten Hindernisse.
Wird eine Bewegung zu einer Position angefordert, kann der generierte Pfad auf potenzielle Kollisionen mit Objekten der Szene überprüft werden.
\subsubsection{Vorschläge für Übungen in der Lehre}
Um die Verwendung des OMX in der Lehre näher zu bringen gibt es in dem für diese Arbeit entstandenen Package mehrere Stellen, an denen angesetzt werden kann.\\
Grundsätzlich lassen sich diese in folgende Bereiche aufteilen:
\begin{itemize}
\item \ac{ROS2}
\item Domäne und Problem in PDDL
\item ...
\end{itemize}
Für jeden dieser Bereiche besteht die Möglichkeit die Aufgaben in die folgenden Richtungen zu führen:
\begin{itemize}
\item Fehlersuche mit sinnvoll platzierten Fehlern
\item Erkennen und Korrigieren weggelassener Abschnitte, welche durch das Verstehen des Codes erkannt werden können
\item die Anpassung des Codes an ein ähnliches Szenario
\end{itemize}
Ein Beispiel für die Domäne ist, den Effekt der \verb|MOVE-GRIPPER| Aktion, der die Startposition löscht erst am Ende der Aktion auszuführen, was dazu führt, dass zum Start des Plans alle nötigen Bewegungen gleichzeitig ausgeführt werden.\\
Eine Aufgabe für das Schreiben/ Anpassen eines Problems in PDDL, welche sich anbietet, ist eine vorgegeben Problemdatei, welche nur die grundlegende Struktur über Stapel und Positionen enthält so zu erweitern, dass eine von mehreren möglichen, vorgegebenen Varianten von gestapelten Blöcken beschreibt.
Die Aufgabe kann dadurch erleichtert werden, dass das gegebene Problem bereits eine bestimmte Stapelung von Blöcken darstellt und nur angepasst werden muss.\\
Für alle Aufgaben die PDDL betreffen bietet es sich an, VS-Code mit dem PDDL-Plugin zu nutzen um schnelle Iterationen und eine Visualisierung der Pläne zu ermöglichen.