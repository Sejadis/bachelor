\section{Einleitung} \label{einleitung}

Roboter sind ein wichtiger Bestandteil der Forschung und der Lehre.
Auch im Fachbereich Informatik und Medien der Technischen Hochschule Brandenburg (THB) sind Roboter Teil und Kern mehrerer Module.
Aktuell werden das an der THB entwickelte AKSEN-Board\footnote{http://ots.th-brandenburg.de/aksen-controller-fur-reaktive-roboter.html} in Kombination mit LEGO sowie Pioneer 2/3 Roboter\footnote{http://ots.th-brandenburg.de/pioneer2-und-pioneer3roboter.html} genutzt.
Zukünftig sollen auch der \acs{ROS2} basierte Roboter TurtleBot3 Waffle Pi\footnote{https://emanual.robotis.com/docs/en/platform/turtlebot3/overview/} mit dem Greifarm OpenMANIPULATOR-X genutzt werden.\\

Im Rahmen dieser Bachelorarbeit soll der Greifarm OpenMANIPULATOR-X der Firma ROBOTIS in Betrieb genommen werden.
Es soll ein Überblick über die grundsätzlichen Vorgänge und Prozesse, bei dessen Nutzung gegeben sowie die Möglichkeiten der Steuerung erprobt werden.
Weiterhin soll die Steuerung mittels Handlungsplanung ermöglicht werden.
Hierzu ist der Stand der Forschung auf dem Gebiet der automatischen Handlungsplanung darzustellen.
Ein gewähltes Planungsverfahren ist zu implementieren und an einem praktischen Beispiel zu demonstrieren.
Für den gewählten Planer wurden verschiedene Szenarien getestet, die die Möglichkeiten und Grenzen zeigen.
Es soll ein \acs{ROS2}-Package erstellt werden, mithilfe dessen der Einstieg in die Nutzung des OpenMANIPULATOR-X vereinfacht werden soll.
Abschließend sollen Möglichkeiten der Einbindung in den Lehrbetrieb gezeigt werden.

