\section{Einleitung} \label{einleitung}
Dieser Teil der Arbeit sollte folgende Inhalte haben:

\begin{itemize}
\item Einführung in die Problemstellung
\item Motivation und Herleitung des Themas
\item Aufbau der Arbeit
\end{itemize}

Hinweis:
Es hat sich als hilfreich erwiesen, die Einleitung mit der Zusammenfassung bzw.\ dem Abstract und der Schlussfolgerung zu vergleichen.
Damit stellt man sicher, dass diese inhaltlich im Bezug auf Zielsetzung und Motivation übereinstimmen.
Der Umfang sollte ca.\ 5 \% der gesamten Arbeit betragen.
\\
\\
Roboter sind ein wichtiger Bestandteil der Forschung und der Lehre.
Auch im Fachbereich Informatik und Medien der Technischen Hochschule Brandenburg (THB) sind Roboter Teil und Kern mehrerer Module.
Aktuell werden das an der THB entwickelte AKSEN-Board\footnote{http://ots.th-brandenburg.de/aksen-controller-fur-reaktive-roboter.html} in Kombination mit LEGO sowie Pioneer 2/3 Roboter\footnote{http://ots.th-brandenburg.de/pioneer2-und-pioneer3roboter.html} genutzt.
Zukünftig sollen auch der \acs{ROS2} basierte Roboter TurtleBot3 Waffle Pi\footnote{https://emanual.robotis.com/docs/en/platform/turtlebot3/overview/} mit dem Greifarm OpenMANIPULATOR-X genutzt werden.\\
In dieser Arbeit soll, speziell für den OMX, ein erster Überblick über die grundsätzlichen Vorgänge und Prozesse bei dessen Nutzung gegeben werden.\\

Im Rahmen der Bachelorarbeit soll der Greifarm OpenMANIPULATOR-X der Firma Robotis in Betrieb genommen sowie die Möglichkeiten der Steuerung erprobt werden.
Weiterhin soll die Steuerung mittels Handlungsplanung ermöglicht werden.
Hierfür sind bestehende Bibliotheken und Frameworks zu evaluieren und ein ausgewähltes zu implementieren und an einem praktischen Beispiel zu demonstrieren.
Für das gewählte Framework oder den gewählten Planer wurden verschieden Szenarien getestet, die die Möglichkeiten und Grenzen zeigen.
Es soll ein \acs{ROS2}-Package erstellt werden, mithilfe dessen der Einstieg in die Nutzung des OpenMANIPULATOR-X vereinfacht werden soll.
Abschließend sollen Möglichkeiten der Einbindung in den Lehrbetrieb gezeigt werden.

